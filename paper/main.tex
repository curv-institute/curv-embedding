% Stability-Driven Chunking for Vector Database Embeddings
% CURV Institute

\documentclass[11pt,a4paper]{article}
\usepackage[utf8]{inputenc}
\usepackage{amsmath,amssymb}
\usepackage{graphicx}
\usepackage{booktabs}
\usepackage{hyperref}
\usepackage{xcolor}

% Placeholder macro for generated content
\newcommand{\placeholder}[1]{\textcolor{red}{[#1]}}

\title{Stability-Driven Chunking for Vector Database Embeddings}

\author{CURV Institute}

\date{\today}

\begin{document}

\maketitle

\begin{abstract}
\placeholder{Abstract to be generated from experimental results.}
\end{abstract}

\section{Introduction}

\placeholder{Introduction referencing PRH (arXiv:2405.07987).}

\section{Method}

\subsection{Cut-Score Formulation}

\placeholder{Cut-score algorithm description.}

\subsection{Offline Chunking}

\placeholder{Offline chunking procedure.}

\subsection{Streaming Chunking}

\placeholder{Streaming chunking procedure.}

\section{Experimental Setup}

\placeholder{Dataset, embedding model, evaluation protocol.}

\section{Results}

\subsection{Embedding Drift}

\placeholder{Drift results.}

% Include generated table
% \begin{tabular}{lrrrrrr}
\toprule
Method & Count & Mean & Std & P50 & P90 & P99 \\
\midrule
Baseline & 0 & 0.0000 & 0.0000 & 0.0000 & 0.0000 & 0.0000 \\
Stability-driven & 0 & 0.0000 & 0.0000 & 0.0000 & 0.0000 & 0.0000 \\
\bottomrule
\end{tabular}

\subsection{Neighbor Churn}

\placeholder{Churn results.}

% \begin{tabular}{lrrrrrr}
\toprule
Method & Updates & Mean & Std & P50 & P90 & Max \\
\midrule
Baseline & 0 & 0.0000 & 0.0000 & 0.0000 & 0.0000 & 0.0000 \\
Stability-driven & 0 & 0.0000 & 0.0000 & 0.0000 & 0.0000 & 0.0000 \\
\bottomrule
\end{tabular}

\subsection{Retrieval Stability}

\placeholder{Overlap results under reformulation.}

% \begin{tabular}{lrrr}
\toprule
Method & k=10 & k=50 & k=100 \\
\midrule
Baseline & 0.0000 & 0.0000 & 0.0000 \\
Stability-driven & 0.0000 & 0.0000 & 0.0000 \\
\bottomrule
\end{tabular}

\subsection{Maintenance Cost}

\placeholder{Re-embedding and rebuild metrics.}

\section{Discussion}

\placeholder{Trade-offs, limitations, relation to PRH.}

\section{Conclusion}

\placeholder{Summary consistent with measured results.}

\section*{Acknowledgments}

These diagnostics were developed as part of a broader internal CURV Institute representation-first framework; only the operational components relevant to the experiments are used here.

\bibliographystyle{plain}
\bibliography{references}

\appendix

% Artifact Appendix for Reproducibility

\section{Artifact Appendix}

\subsection{Repository}

All code, data generation scripts, and experiment manifests are available at:
\begin{center}
\url{https://github.com/curv-institute/curv-embedding}
\end{center}

\subsection{Requirements}

\begin{itemize}
  \item Python $\geq$ 3.12
  \item Dependencies declared via PEP 723 inline headers
  \item Executable via \texttt{uv run}
\end{itemize}

\subsection{Reproducing Results}

\begin{verbatim}
git clone https://github.com/curv-institute/curv-embedding
cd curv-embedding
uv run scripts/reproduce.py --run-name <run_name>
\end{verbatim}

\subsection{Manifest Format}

Each run produces:
\begin{itemize}
  \item \texttt{manifest.json}: Configuration snapshot and seeds
  \item \texttt{metrics.jsonl}: Raw metric measurements
  \item \texttt{summary.json}: Aggregated statistics
  \item \texttt{meta.sqlite}: Full chunk and event audit log
\end{itemize}

\subsection{Compute Resources}

All experiments were run on a single workstation with an Intel Core i7 processor and 32GB RAM. Total runtime for the full experiment suite is under 10 minutes.


\end{document}
